% \documentclass[11pt,UTF8]{beamer}
\documentclass[11pt,UTF8]{ctexbeamer}

\usepackage[utf8]{inputenc}
\usepackage[T1]{fontenc}

\usetheme{Cuerna}
\usecolortheme{brick}
\graphicspath{{figure/}}
% \usepackage{xeCJK}
% Windows中文字体
% \setCJKsansfont{KaiTi}%

\logo{\includegraphics[width=1.7cm]{cumtb}}

\title{Model-Based Geostatistics}
\subtitle{Generalized Linear Geostatistical Models and Its Applications}
\author{Xiangyun~Huang \and XXX \\
\texttt{XXX@outlook.com} }
\institute[China University of Mining and Technology,Beijing]
{
Department of Statistics\\ % Computational mathematics
China University of Mining and Technology,Beijing
}
\date{\today}

\begin{document}


\begin{frame}
\titlepage
\end{frame}

\begin{frame}
\frametitle{Outline}
\begin{itemize}
  \item Motivation
  \item Introduction 
  \item Geostatistical model
  \item Real data analysis (Applications)
  \item Discussion
\end{itemize}
\end{frame}

\section{Introduction}
\subsection{Motivation}
  \begin{frame}
    \frametitle{Title of the slide}
    \framesubtitle{Subtitle}
    Some text on top of this slide
    \[
    \int_{0}^{\infty} \frac{5x^2}{\sqrt{a+b}}\, dx
    \]
    \begin{itemize}
    \item[\checkmark] Fist item in the list
    \item Second item
    \item Another item
    \item And finally, the last item
  \end{itemize}
  \end{frame}
 
\begin{frame}{预测任务}
\begin{itemize}
\item 重金属浓度最高的位置 (最大值)
\item 重金属浓度超过给定值的区域  (区间估计)
\item 重金属在某个区域的含量 (积分)
\end{itemize}
\end{frame}

\begin{frame}
\begin{figure}
\centering
\includegraphics[width=.7\textwidth]{cumtb}
\end{figure}
\end{frame} 
 
 
\end{document}

